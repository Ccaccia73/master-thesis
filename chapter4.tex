\chapter{Design of a Mitosis Detection algorithm}
\label{chapter4}
\thispagestyle{empty}

\begin{quotation}
{\footnotesize
\noindent \emph{``Ab uno\\ disces omnis''}\\
\noindent (Learn everything from one)
\begin{flushright}
Publius Vergilius Maro (Aeneis II, 65-66)
\end{flushright}
}
\end{quotation}

\vspace{0.5cm}

%\noindent In questa sezione si spiega come \`e stato affrontato il problema concettualmente, la soluzione logica che ne \`e seguita senza la documentazione.

\section{Structure}

General structure of a Mitosis Detection algorithm.

\vspace{0.5cm}

\section{Feature Extraction}

(Qui o prima bisogna esplicitare che utilizziamo un subset di immagini)

\section{Classifiers}









\section{Classification Assessment}

\subsection{Algorithms}






In our work we focused on two types of classifiers: \textit{Support Vector Machines} and \textit{Random Forests}
which are widely used in computer vision classification problems ( e.g. \cite{mitosisDetectionLearningBased} and \cite{randForests04}).
We also mention \Glspl{CNN} because they played a relevant role in the definition of our dataset [REF].


 

The role of features and classifiers

\subsection{Feature Extraction}

Curse of dimensionality and PCA

[SNIPPET]
In most computer vision applications it is not sufficient to extract only one type of feature to obtain the relevant information from the image data.
Instead two or more different features are extracted, resulting in two or more feature descriptors at each image point.
A common practice is to organize the information provided by all these descriptors as the elements of one single vector,
commonly referred to as a feature vector. The set of all possible feature vectors constitutes a feature space.
A common example of feature vectors appears when each image point is to be classified as belonging to a specific class.
Assuming that each image point has a corresponding feature vector based on a suitable set of features,
meaning that each class is well separated in the corresponding feature space, the classification of each image point can be done using standard classification method.


\subsection{Humans}

Experience, agreement...

\vspace{0.5cm}



\section{Benchmarks}
\label{ch3:Bench}
