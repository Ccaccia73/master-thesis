\chapter{State of the art}
\label{chapter2}
\thispagestyle{empty}

\begin{quotation}
{\footnotesize
\noindent{\emph{``Rem tene, verba sequentur''}\\
(Know the subject, the words will follow)
}
\begin{flushright}
Marcius Porcius Cato Censorius
\end{flushright}
}
\end{quotation}
\vspace{0.5cm}

%\noindent Nella seconda sezione si riporta lo stato dell'arte del settore, un inquadramento dell'area di ricerca orientato a portare il lettore all'interno della problematica affrontata. Bisogna dimostrare di conoscere le cose fatte fino ad ora in questo campo e il perch\'e si sia reso necessario lo svolgimento di questo lavoro. Questa sezione deve essere grondante di citazioni bibliografiche \cite{Adaptative}.

\section{Detection Problems}
General overview of the detection problems.

\vspace{0.5cm}

\section{Feature extraction problems and classifiers}
General description of the feature extraction based approach and classification

\vspace{0.5cm}

\section{Mitosis Detection}
Some biological background:
\begin{itemize}
\item What is a mitosis
\item Why it is important in breast cancer classification
\item Methods of classification of breast cancer
\end{itemize}

\section{Benchmarks}

\vspace{0.5cm}

\subsection{Humans}
Agreement between different histologists

\subsection{Algorithms}
Benchmarking of different detection algorithms and comparison with human performance.