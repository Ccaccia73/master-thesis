\chapter{Problem Definition}
\label{chapter3}
\thispagestyle{empty}

\begin{quotation}
{\footnotesize
\noindent{\emph{``\greek{p'antec >'anjrwpoi to~u e>id'enai >or'egontai f'usei}''}\\
(All men naturally desire knowledge)}
\begin{flushright}
\greek{>Aristot'elhc}(Aristotle, Met. 1.980a)
\end{flushright}
}
\end{quotation}


%\noindent In questa sezione si deve descrivere l'obiettivo della ricerca, le problematiche affrontate ed eventuali definizioni preliminari nel caso la tesi sia di carattere teorico.


\section{Bacground}

The purpose of automating the mitosis detection problems requires the definition 

\begin{tikzpicture}[node distance = 2cm, auto]
    % Place nodes
    \node [block] (init) {initialize model};
    \node [cloud, left of=init] (expert) {expert};
    \node [cloud, right of=init] (system) {system};
    \node [block, below of=init] (identify) {identify candidate models};
    \node [block, below of=identify] (evaluate) {evaluate candidate models};
    \node [block, left of=evaluate, node distance=3cm] (update) {update model};
    \node [decision, below of=evaluate] (decide) {is best candidate better?};
    \node [block, below of=decide, node distance=3cm] (stop) {stop};
    % Draw edges
    \path [line] (init) -- (identify);
    \path [line] (identify) -- (evaluate);
    \path [line] (evaluate) -- (decide);
    \path [line] (decide) -| node [near start] {yes} (update);
    \path [line] (update) |- (identify);
    \path [line] (decide) -- node {no}(stop);
    \path [line,dashed] (expert) -- (init);
    \path [line,dashed] (system) -- (init);
    \path [line,dashed] (system) |- (evaluate);
\end{tikzpicture}












\vspace{0.5cm}



\section{From Detection to Classification}

The process of detection and classification....

\vspace{0.5cm}

\section{Definition of Classification}

Definition of classification:
\begin{itemize}
\item input
\item output
\item classes
\end{itemize}

\vspace{0.5cm}

\section{Classification Assessment}

\subsection{Algorithms}






In our work we focused on two types of classifiers: \textit{Support Vector Machines} and \textit{Random Forests}
which are widely used in computer vision classification problems ( e.g. \cite{mitosisDetectionLearningBased} and \cite{randForests04}).
We also mention \Glspl{CNN} because they played a relevant role in the definition of our dataset [REF].


 

The role of features and classifiers

\subsection{Feature Extraction}

Curse of dimensionality and PCA










[SNIPPET]
In most computer vision applications it is not sufficient to extract only one type of feature to obtain the relevant information from the image data.
Instead two or more different features are extracted, resulting in two or more feature descriptors at each image point.
A common practice is to organize the information provided by all these descriptors as the elements of one single vector,
commonly referred to as a feature vector. The set of all possible feature vectors constitutes a feature space.
A common example of feature vectors appears when each image point is to be classified as belonging to a specific class.
Assuming that each image point has a corresponding feature vector based on a suitable set of features,
meaning that each class is well separated in the corresponding feature space, the classification of each image point can be done using standard classification method.






\subsection{Humans}

Experience, agreement...

\vspace{0.5cm}

\section{Performance}

Definition of performance


[SNIPPET]
The general appearance of a mitosis results in the fact that automatically detecting mitoses is very challenging.
Different to other pattern recognition tasks, mitotic cells essentially are irregular shape objects. As a result, there
is no simple way of extracting the features of mitotic cells.
Benchmarking of different detection algorithms and comparison with human performance.


\section{Benchmarks}
\label{ch3:Bench}

\vspace{0.5cm}

\subsection{Humans}
\label{ch3:humans}


Agreement between different histologists

\subsection{Algorithms}

The technique of \textit{thresholding} is often used to analyze the performance of an algorithm that outputs probabilities in 

