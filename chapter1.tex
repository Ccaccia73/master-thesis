\setcounter{page}{1}
\pagenumbering{arabic}

\chapter{Introduction}
\label{Introduction}
\thispagestyle{empty}

\begin{quotation}
{\footnotesize
\noindent{\emph{``Quote 1''}
}
\begin{flushright}
Author 1
\end{flushright}
}
\end{quotation}
\vspace{0.5cm}

%\noindent L'introduzione deve essere atomica, quindi non deve contenere n\`e sottosezioni n\`e paragrafi n\`e altro. Il titolo, il sommario e l'introduzione devono sembrare delle scatole cinesi, nel senso che lette in quest'ordine devono progressivamente svelare informazioni sul contenuto per incatenare l'attenzione del lettore e indurlo a leggere l'opera fino in fondo. L'introduzione deve essere tripartita, non graficamente ma logicamente:


%\section{Inquadramento generale}
%La prima parte contiene una frase che spiega l'area generale dove si svolge il lavoro; una che spiega la sottoarea pi\`u specifica dove si svolge il lavoro e la terza, che dovrebbe cominciare con le seguenti parole ``lo scopo della tesi \`e \dots'', illustra l'obbiettivo del lavoro. Poi vi devono essere una o due frasi che contengano una breve spiegazione di cosa e come \`e stato fatto, delle attivit\`a  sperimentali, dei risultati ottenuti con una valutazione e degli sviluppi futuri. La prima parte deve essere circa una facciata e mezza o due

\emph{First part topics}

\begin{itemize}
\item Detection problems in Computer Vision and in particular in biomedical imaging
\item Relation between detection and classification
\item Mitosis Detection as a component in breast cancer assessment
\item Machine Learning used to automate the mitotic count task
\item The validation problem:
	\begin{itemize}
	\item from clinical point of view
	\item from ML point of view
	\end{itemize}
\end{itemize}

\vspace{0.5cm}

%\section{Inquadramento generale}
%La prima parte contiene una frase che spiega l'area generale dove si svolge il lavoro; una che spiega la sottoarea pi\`u specifica dove si svolge il lavoro e la terza, che dovrebbe cominciare con le seguenti parole ``lo scopo della tesi \`e \dots'', illustra l'obbiettivo del lavoro. Poi vi devono essere una o due frasi che contengano una breve spiegazione di cosa e come \`e stato fatto, delle attivit\`a  sperimentali, dei risultati ottenuti con una valutazione e degli sviluppi futuri. La prima parte deve essere circa una facciata e mezza o due

\emph{Second part topics}

\begin{itemize}
\item General overview of the work: automatic Mitosis Detection in breast cancer histological images and comparison of the performances between humans and algorithms.
	\begin{itemize}
	\item some literature
	\item specificity of this work
	\item achievements
	\item research directions
	\end{itemize}
\end{itemize}


%\section{Breve descrizione del lavoro}
%La seconda parte deve essere una esplosione della prima e deve quindi mostrare in maniera pi\`u esplicita l'area dove si svolge il lavoro, le fonti bibliografiche pi\`u importanti su cui si fonda il lavoro in maniera sintetica (una pagina) evidenziando i lavori in letteratura che presentano attinenza con il lavoro affrontato in modo da mostrare da dove e perch\'e \`e sorta la tematica di studio. Poi si mostrano esplicitamente le realizzazioni, le direttive future di ricerca, quali sono i problemi aperti e quali quelli affrontati e si ripete lo scopo della tesi. Questa parte deve essere piena (ma non grondante come la sezione due) di citazioni bibliografiche e deve essere lunga circa 4 facciate.


\emph{Third part topics}

\begin{itemize}
\item Structure of the work
	\begin{itemize}
	\item Section 1: state of the art...
	\item Section 2: approach to the problem and model
	\item Section 3: design of a mitosis detection algorithm
	\item Section 4: design of a user study
	\item Section 5: experimental results
	\item Section 6: Conclusions
	\item Appendixes: implementation details
	\end{itemize}
\end{itemize}

%\section{Struttura della tesi}
%La terza parte contiene la descrizione della struttura della tesi ed \`e organizzata nel modo seguente.
%``La tesi \`e strutturata nel modo seguente.

%Nella sezione due si mostra \dots

%Nella sez. tre si illustra \dots

%Nella sez. quattro si descrive \dots

%Nelle conclusioni si riassumono gli scopi, le valutazioni di questi e le prospettive future \dots

%Nell'appendice A si riporta \dots (Dopo ogni sezione o appendice ci vuole un punto).''

%I titoli delle sezioni da 2 a M-1 sono indicativi, ma bisogna cercare di mantenere un significato equipollente nel caso si vogliano cambiare. Queste sezioni possono contenere eventuali sottosezioni.
 

%\Gls{naiive} people don't know about alternative \gls{computer} operating systems: \glspl{Linux}, BSDs and GNU/Hurd.