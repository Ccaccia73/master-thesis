\newpage
\chapter*{Sommario}

\addcontentsline{toc}{chapter}{Sommario}

%Il sommario deve contenere 3 o 4 frasi tratte dall'introduzione di cui la prima inquadra l'area dove si svolge il lavoro (eventualmente la seconda inquadra la sottoarea pi\`u specifica del lavoro), la seconda o la terza frase dovrebbe iniziare con le parole ``Lo scopo della tesi \`e \dots'' e infine la terza o quarta frase riassume brevemente l'attivit\`a  svolta, i risultati ottenuti ed eventuali valutazioni di questi.

\vspace{1.3cm}
%\noindent NB: se il relatore effettivo \`e interno al Politecnico di Milano nel frontesizo si scrive Relatore, se vi \`e la collaborazione di un altro studioso lo si riporta come Correlatore come sopra. Nel caso il relatore effettivo sia esterno si scrive Relatore esterno e poi bisogna inserire anche il Relatore interno. Nel caso il relatore sia un ricercatore allora il suo Nome COGNOME dovr\`{a} essere preceduto da Ing. oppure Dott., a seconda dei casi.


Il presente lavoro di tesi si colloca nel contesto dell'apprendimento automatico, in particolare 
nel campo della classificazione automatizzata di immagini istologiche.\\
In numerose tipologie di carcinoma, l'identificazione dello stadio di avanzamento della malattia gioca un ruolo fondamentale
per la selezione delle cure migliori e nella riduzione del tasso di mortalit\`{a}. Attualmente, la classificazione
dello stadio di un tumore \`{e} eseguita manualmente dall'istologo su campioni di tessuto analizzati al microscopio.
L'applicazione di tecniche di \textit{computer vision} e di \textit{machine learning} possono portare a numerosi benefici in termini di
tempo e qualit\`{a} delle analisi eseguite.

\vspace{0.3cm}

Il conteggio delle mitosi in un'immagine istologica costituisce uno dei criteri pi\`{u} rigorosi per la classificazione dei tumori e
si rivela essere un'attivit\`{a} complessa anche per un occhio molto allenato. Per tale motivo, l'identifica-zione automatica delle 
mitosi \`{e} un tema di ricerca molto interessante e pu\`{o} essere visto come un caso di apprendimento con supervisione, in cui un classificatore 
ha a disposizione un insieme di immagini di esempio gi\`{a} etichettate e da queste deve inferire dei criteri per classificarne altre.

\vspace{0.3cm}

In generale, da un punto di vista applicativo, l'interesse \`{e} focalizzato sulle prestazioni di un tale sistema: l'obiettivo
consiste nell'ottenere dei risultati uguali o migliori rispetto a quelli ottenuti dall'occhio umano esperto.

\vspace{0.3cm}

In questo lavoro ci poniamo nell'ottica di chi progetta un algoritmo di apprendimento. In questo contesto,
confrontare un algoritmo con un pato-logo esperto non fornisce un'informazione utile: infatti, un medico esperto ha avuto accesso,
durante la sua esperienza lavorativa, ad un insieme di dati di esempio e di linee guida di gran lunga superiori rispetto ad un normale
insieme di \textit{training} di un algoritmo di apprendimento.


\vspace{0.3cm}

Una prestazione inadeguata dell'algoritmo pu\`{o} essere causata dalla sua scarsa capacit\`{a} di identificazione o dalla
mancanza di un numero sufficiente di dati di esempio.
Tramite i risultati delle nostre analisi rispondiamo a tale quesito nel contesto del conteggio di mitosi in immagini istologiche, focalizzandoci sul problema di classificazione.
Abbiamo estratto dei campioni da immagini prese da un \textit{dataset} pubblico e classificato da un patologo esperto.
A questo insieme abbiamo applicato una serie di classificatori ed abbiamo implementato un'interfaccia web per far eseguire 
la classificazione a diversi utenti.

\vspace{0.3cm}

\noindent Questa tesi porta due contributi principali:
\begin{itemize}
 \item Un confronto quantitativo tra l'accuratezza degli algoritmi di \textit{mitosis detection} allo stato dell'arte e la performance di umani che non ope-rano quotidiananmente nel medesimo campo
 e che vengono posti nelle medesime condizioni. L'obiettivo consiste nel determinare le cause delle differenze riscontrate.
 \item Uno studio dettagliato del problema di classificazione sotteso a tale problema di \textit{detection}, quantificando l'importanza di vari fattori quali:
 scelta dell'algoritmo di classificazione, scelta delle \textit{features}, dimensione del \textit{training set}, dimensione dell'immagine ed altri.
\end{itemize}

La struttura della tesi segue le tematiche qui descritte, introducendo il problema nel Capitolo \ref{Introduction} e fornendo un quadro dello stato dell'arte nel Capitolo \ref{chapter2}.
Il lavoro prosegue definendo le caratteristiche richieste ad un algoritmo di \textit{mitosis detection} ed illustrando le misure di performance pi\`{u} significative, evidenziando
quelle adottate per i confronti dei risultati (Capitolo \ref{chapter3}). Il Capitolo \ref{chapter4} illustra nel dettaglio gli algoritmi di classificazione adottati e le tipologie
di feature adottate per descrivere i campioni. Il Capitolo \ref{chapter5} descrive l'interfaccia web appositamente realizzata per la raccolta dei dati delle classificazioni 
eseguite dagli umani, mentre il Capitolo \ref{chapter6} \`{e} dedicato all'analisi dei risutati ed ai confronti. Il Capitolo \ref{chapter7} riporta le conclusioni del lavoro
effettuato, che si possono cos\`{\i} riassumere: gli esiti del confronto fra i risultati degli utenti e quelli dei classificatori automatici evidenziano che la prestazione degli
algoritmi \`{e} migliore di quella degli umani (risultato coerente con altri studi analizzati) e che quindi la dimensione del dataset
\`{e} un elemento determinante per la determinazione delle prestazioni di un algoritmo di identificazione.
