\chapter{Design of a User Study}
\label{chapter5}
\thispagestyle{empty}

\begin{quotation}
{\footnotesize
\noindent{\emph{``\greek{`p'antwn qrhm'atwn m'etron', >'anjrwpon e>~inai, 't~wn m`en >'ontwn <wc >'esti, t~wn d`e m`h >'ontwn <wc o>uk >'estinv.'}''}\\
(man is \textquotedblleft the measure of all things, of the existence of the things that are and the non-existence of the things that are not.\textquotedblright)}
\begin{flushright}
\greek{Pl'atwn}(Plato, Theaet. 152a)
\end{flushright}
}
\end{quotation}

% ὄντων ὡς ἔστι, τῶν δὲ μὴ ὄντων ὡς οὐκ ἔστιν.’

\vspace{0.5cm}

%\noindent Si mostra il progetto dell'architettura del sistema con i vari moduli.

\section{Test Design}

The problem of detecting mitosis can be cast as a problem of classifying image
patches. In fact, most detection algorithms are based on classifiers which map
an image patch to the probability that a mitosis appears at its center; once such
classifier is known, the detection problem is solved by applying it on a sliding
window over the input image, or to a set of candidate patches identified in a
previous step.
The classification task can be presented to an user through a very simple and immediate interaction mechanism: in fact, a single decision is required for each
patch. In contrast, detection would require a more complicated interaction with
users. For this reason, we focus on the classification subproblem in the following.
For a given sample, input is given in form of an image patch with size 100 × 100
px: such size completely contains the image of the cell, and most algorithms
(FIXME) considered in the following only use data from a smaller window. The
task is to map each patch to one of two classes: C1) the image contains a mitosis
at its center; C0) the image does not contain a mitosis anywhere. There are no
samples in which a mitosis is visible off-center.


\subsection{Dataset}

(NB: il set di immagini usate deve esser già stato descritto)

\subsection{User Interface}

Description of the website used to collect data from users.

\vspace{0.5cm}

\section{Data collection}

Description of the data collected by the website